% https://nymity.ch/tikz/

\documentclass[hyperref={pdfpagelabels=true},table,dvipsnames,14pt,aspectratio=169]{beamer}
\usetheme{Boadilla}
\setbeamertemplate{navigation symbols}{}

\usepackage{amsthm, amsfonts, amssymb, amsmath, graphicx}
\usepackage{amsmath}
\usepackage{mathtools}
\usepackage{amsfonts, txfonts}
\usepackage{wasysym}
\usepackage{xcolor}
\usepackage{caption}
\usepackage{comment, pdfpcnotes}
\usepackage{cryptocode}
\urlstyle{same}
\usepackage{framed}
\FrameSep5pt

\usepackage{graphicx}
\graphicspath{ {images/} }

\usepackage{tikz}
\usepackage{tikzpeople}
\usepackage{pgfplots}
\usetikzlibrary{arrows}

\usetikzlibrary{positioning}
\usetikzlibrary{calc}
\usetikzlibrary{shapes}
\usetikzlibrary{shapes.callouts}

\usepackage{capt-of}
\beamertemplatenavigationsymbolsempty
\setbeamercovered{transparent}

\newcommand{\sigagg}{\mathcal{SA}}
\newcommand{\random}{\overset{\$}{\leftarrow}}

% Method to add an item that will appear in every slide
%\addtobeamertemplate{footline}{%
%
%  \tikz[remember picture, overlay] \fill[blue] (current page.south east)
%    rectangle ++(1cm,2cm);
%
%}{}

\pgfdeclareimage[height=7ex]{crysplogo}{crysplogo}
%\pgfdeclareimage[height=9ex, width=30ex]{cscslogo}{CSCSlogo}
\pgfdeclareimage[height=7ex]{zfndlogo}{zfndlogo}

\titlegraphic{
\vspace{-1ex}
%\raisebox{1ex}{
\pgfuseimage{crysplogo}~~~
    \hspace{1em}
    \hspace{1em}
%\raisebox{0.5ex}{\pgfuseimage{cscslogo}}
%    \hspace{1em}
%    \hspace{1em}
\pgfuseimage{zfndlogo}~~~
%}
}


\pgfdeclarelayer{background}
\pgfdeclarelayer{backbackground}
\pgfdeclarelayer{foreground}
\pgfsetlayers{backbackground,background,main,foreground}   %% some additional layers for demo

\usetikzlibrary{shapes,decorations.shapes}
\tikzset{>=latex}



\makeatletter
\pgfdeclareshape{document}{
\inheritsavedanchors[from=rectangle] % this is nearly a rectangle
\inheritanchorborder[from=rectangle]
\inheritanchor[from=rectangle]{center}
\inheritanchor[from=rectangle]{north}
\inheritanchor[from=rectangle]{south}
\inheritanchor[from=rectangle]{west}
\inheritanchor[from=rectangle]{east}
% ... and possibly more
\backgroundpath{% this is new
% store lower right in xa/ya and upper right in xb/yb
\southwest \pgf@xa=\pgf@x \pgf@ya=\pgf@y
\northeast \pgf@xb=\pgf@x \pgf@yb=\pgf@y
% compute corner of ‘‘flipped page’’
\pgf@xc=\pgf@xb \advance\pgf@xc by-10pt % this should be a parameter
\pgf@yc=\pgf@yb \advance\pgf@yc by-10pt
% construct main path
\pgfpathmoveto{\pgfpoint{\pgf@xa}{\pgf@ya}}
\pgfpathlineto{\pgfpoint{\pgf@xa}{\pgf@yb}}
\pgfpathlineto{\pgfpoint{\pgf@xc}{\pgf@yb}}
\pgfpathlineto{\pgfpoint{\pgf@xb}{\pgf@yc}}
\pgfpathlineto{\pgfpoint{\pgf@xb}{\pgf@ya}}
\pgfpathclose
% add little corner
\pgfpathmoveto{\pgfpoint{\pgf@xc}{\pgf@yb}}
\pgfpathlineto{\pgfpoint{\pgf@xc}{\pgf@yc}}
\pgfpathlineto{\pgfpoint{\pgf@xb}{\pgf@yc}}
\pgfpathlineto{\pgfpoint{\pgf@xc}{\pgf@yc}}
}
}
\makeatother




\title[FROST]{FROST: \\ Flexible Round-Optimized \\ Schnorr Threshold Signatures}
\author[Chelsea Komlo, Ian Goldberg]{\textbf{Chelsea Komlo}$^{1,2}$
\hspace{2em} Ian Goldberg$^1$}
\institute[]{\small $^1$ University of Waterloo\hspace{4em}$^2$ Zcash
Foundation}
\date[October 2020]{ \small University of College London, December 2020}

\begin{document}
\setbeamertemplate{itemize items}[triangle]

\tikzstyle{doc}=[%
draw,
thick,
align=center,
color=black,
shape=document,
minimum width=12mm,
minimum height=15.2mm,
shape=document,
inner sep=2ex,
]

\begin{frame}
        \thispagestyle{empty}
        \maketitle
\end{frame}

\begin{frame}
  \frametitle{Threshold Secret Sharing }
  \begin{itemize}
    \item<1-> Partitions a secret
      among a set of participants, such that recovering/using the
      secret requires cooperation among a threshold number of participants.
    \item[]~
    \item<2-> Shamir secret sharing is the most well-known algorithm and what
      FROST builds upon.
    \item[]~
    \item<3-> $n$ represents the \emph{total} number of allowed participants;
    $t$ the threshold.
  \end{itemize}
\end{frame}

\begin{frame}
  \frametitle{Why use Secret Sharing?}
  \begin{itemize}
    \item<1-> Disaster recovery.
    \item[]~
    \item<2-> Raise the bar for an adversary.
    \item[]~
    \item<3-> Partition trust.
  \end{itemize}
\end{frame}

\begin{frame}{$(t,n)$-Threshold Secret Sharing }{$(2,3)$-Threshold Scheme Example}
  \hspace{3ex}
    \begin{tikzpicture}[scale=.6]
      \only<3->{\node[person,shirt=green,female,minimum size=1.1cm] (p2) at
      (8.5,1) {
        $\includegraphics[width=.075\textwidth]{images/seckey.png}_\text{\bf
        Secret Share 2}$
      };}
        \only<3->{\node[person,shirt=green, female,minimum size=1.1cm] (p6) at
        (12,-3) {
        $\includegraphics[width=.075\textwidth]{images/seckey.png}_\text{\bf
        Secret Share 3}$
        };}
        %\node[doc] (x) at (9.5,2) {};
        \begin{pgfonlayer}{foreground}
          \only<2->{\node[inner sep=0pt] at (18,0.6)
          {\includegraphics[width=.1\textwidth]{fire.png}}};
        \only<4->{\node[ellipse, draw=red, line width = 2pt,
          minimum width = 6cm, minimum height =3cm] at (12.25,-2.25) {}};
    \node<4> [fill=yellow,draw=red,ultra thick, text width=.3\textwidth,anchor=north west,ellipse callout,text centered,callout
  absolute pointer = {(15,1)}] at (18,1) { The full secret
        can be recovered using $t$ out of any of the $n$ participants.};
        \end{pgfonlayer}


        \node[person,shirt=black, monitor, female,minimum size=1.1cm] (p7) at
        (18,1) {
        $\includegraphics[width=.075\textwidth]{images/seckey.png}_\text{\bf
        Secret Share 1}$
      };


      \begin{pgfonlayer}{background}
            \node[circle, minimum size = 6.4cm, color=lightgray,
            fill=lightgray]  (circ) at (12.5,-0.5){};
      \end{pgfonlayer}
    \end{tikzpicture}
\end{frame}

\begin{frame}{$(t,n)$-Threshold Secret Sharing}{$(2,3)$-Threshold Scheme Example}
    \begin{tikzpicture}[scale=.6]
      \only<2->{\node[person,shirt=green,good,female,minimum size=1.1cm] (p2) at
      (6.5,-1) {
        $\includegraphics[width=.075\textwidth]{images/seckey.png}_\text{\bf
        Secret Share 2}$
      };}
        \only<2->{\node[person,shirt=green, good,female,minimum size=1.1cm] (p6) at
        (11,-4) {
        $\includegraphics[width=.075\textwidth]{images/seckey.png}_\text{\bf
        Secret Share 3}$
        };}

        \begin{pgfonlayer}{foreground}
    \node<3> [fill=yellow,draw=red,ultra thick, text width=.3\textwidth,anchor=north west,ellipse callout,text centered,callout
  absolute pointer = {(13.5,1)}] at (18,1) { An adversary must compromise at
          least $t$ participants to control the secret. };
    \node<4> [fill=yellow,draw=red,ultra thick, text width=.3\textwidth,anchor=north west,ellipse callout,text centered,callout
  absolute pointer = {(13.5,1)}] at (18,1) { Requires the assumption that the
          adversary controls fewer than $t$ participants. };
        \end{pgfonlayer}
        \node[person,shirt=black, evil, female,minimum size=1.1cm] (p7) at (13,1) {
        $\includegraphics[width=.075\textwidth]{images/seckey.png}_\text{\bf
        Secret Share 1}$
        };


      \begin{pgfonlayer}{background}
            \node[circle, minimum size = 6.4cm, color=lightgray, fill=lightgray]  (circ) at (9.5,-0.5){};
      \end{pgfonlayer}
    \end{tikzpicture}
\end{frame}


\begin{frame}
  \frametitle{Threshold Signatures: Joint Public \\ Key, Secret-Shared Private Key}

\begin{tikzpicture}
      \begin{pgfonlayer}{foreground}
    \node<3> [fill=yellow,draw=red,ultra thick, text width=.3\textwidth,anchor=north west,ellipse callout,text centered,callout
  absolute pointer = {(9.2,1)}] at (12,1) { The full secret
        is never reconstructed! Participants perform signing using only their secret
        share.};
    \node<4> [fill=yellow,draw=red,ultra thick, text width=.4\textwidth,anchor=north west,ellipse callout,text centered,callout
  absolute pointer = {(9.2,1)}] at (12,1) { The signature issued can look
        identical to a single-party signature, which is attractive for privacy
        and backwards-compatibility reasons.};
      \end{pgfonlayer}
      \begin{pgfonlayer}{main}
      \node[person,shirt=black,female, minimum size=1.1cm] (p2) at (7.5,1) {
        $\includegraphics[width=.075\textwidth]{images/seckey.png}_\text{\bf
        Secret Share 1}$
      };
      \node[person,shirt=black, female, minimum size=1.1cm] (p6) at (9,-2) {
        $\includegraphics[width=.075\textwidth]{images/seckey.png}_\text{\bf
        Secret Share 2}$
      };
      \node[person,shirt=black,female, minimum size=1.1cm] (p7) at (13,1) {
        $\includegraphics[width=.075\textwidth]{images/seckey.png}_\text{\bf
        Secret Share 3}$
      };

      \end{pgfonlayer}

      \begin{pgfonlayer}{backbackground}
      \node[circle, minimum size = 5.4cm, color=lightgray,
        fill=lightgray]  (circ) at (10.5,-0.5){};
      \end{pgfonlayer}
      \begin{pgfonlayer}{background}
          \node[ellipse, draw=red, line width = 2pt,
          minimum width = 9.25cm, minimum height =3cm] at (10,0.5) {\small
          \bf Signing Set};
      \end{pgfonlayer}

       \node[outer sep = 2mm] (pubkey) at (18,-0.5)
       {
         \only<2>{$\includegraphics[width=.15\textwidth]{images/pubkey.png}_\text{\bf
         Public Key (1,2,3)}$  }};

\end{tikzpicture}
\end{frame}

\begin{frame}
  \frametitle{Applications of Threshold Signing}
  \begin{itemize}
    \uncover<1->{\item Issuance of certificates by certificate authorities.}
    \uncover<1->{\item[]~}
    \uncover<2->{\item Distribution of Tor's consensus by directory authorities.}
    \uncover<2->{\item[]~}
    \uncover<3->{\item Authentication of blockchain transactions.}
    \uncover<3->{\item[]~}
    \uncover<4->{\item In general, distributed authentication that requires some minimum
      number of signers.}
  \end{itemize}
\end{frame}

\begin{frame}
  \frametitle{Comparison to Multisignature Schemes}
  \centering

  \begin{itemize}
    \item<1->  A multisignature is a compact representation of $n$ signatures
      over some message.
    \item<2-> Each signer has their own public/private keypair.
    \item<3-> No enforced access structure in the primitive itself.
  \end{itemize}

  \uncover<4-> {\begin{table}[t]
    \centering
    \footnotesize
    \begin{tabular}{c|c|c|c|c }
      & \textbf{$t$ out of $n$?} & \textbf{Dynamic Signing Groups?} &
      \textbf{Single Public Key?} \\
      \hline
      Multisignature & No & Yes& Yes   \\
      Threshold & Yes & No & Yes  \\
    \end{tabular}
  \end{table} }
\end{frame}

\begin{frame}
  \frametitle{Contributions of FROST}

  \begin{itemize}
    \item<1-> Two-round threshold signing protocol, or single-round \\ protocol with
    preprocessing
    \item[]
    \item<2-> Signing operations are secure when performed concurrently,
      improving upon prior similar schemes.
    \item[]
    \item<3-> Signing can be performed with a threshold $t$ number of signers,
      where $t$ can be less than the number of possible signers $n$.
    \item[]
    \item<4-> Secure against an adversary that controls up to $t-1$ signers.
  \end{itemize}
\end{frame}


\begin{frame}
  \frametitle{Tradeoffs Among Constructions}

  \begin{itemize}
    \item<1-> \textbf{Number of Signing Rounds:} Required network rounds to
      generate one signature.
    \item<2-> \textbf{Robust:} Can the protocol complete when participants
      misbehave?
    \item<3-> \textbf{Required Number of Signers:} Can a signature be
    created by just $t$ participants, or are all $n$ needed?
    \item<4-> \textbf{Parallel Secure:} Can signing operations be done in
      parallel without a reduction in security (Drijvers attack)?
  \end{itemize}
\end{frame}

\begin{frame}
  \frametitle{Tradeoffs Among Constructions}
  \centering

  \begin{table}[t]
    \centering
    \footnotesize
    \begin{tabular}{c|c|c|c|c|c}
      & \textbf{Num. Rounds} & \textbf{Robust?} & \textbf{Preprocessing?} & \textbf{Num. Signers}&
      \textbf{Parallel Secure?} \\
      \hline
      \only<1->{Stinson Strobl & 4 & Yes & No & $t$ & Yes \\ }
      \only<2->{Gennaro et al & 2 & No & Yes & $n$ & No \\ }
      \only<3->{FROST & 2 & No & Yes & $t$ & Yes \\ }
    \end{tabular}
\end{table}
\end{frame}

\pnote{
Schnorr signatures are the standard $\Sigma$-protocol proof of
knowledge of the discrete logarithm of $Y$ made non-interactive
with the Fiat-Shamir transform.

}
\begin{frame}
  \centering
\scalebox{0.9} {
\procedure{Single-Party Schnorr Signing and Verification}{
  \textbf{Signer}  \< \< \textbf{Verifier} \\
  \uncover<1-> { (x, Y) \leftarrow \mathit{KeyGen()} \\ }
  \uncover<2-> { \< \sendmessageleft{top=\text{$(m, Y)$}} \< \\ }
  \uncover<3-> { k \random \mathbb{Z}_q} \\
  \uncover<4-> { R = g^k \in \mathbb{G} } \\
  \uncover<5-> { c = H(R, Y, m) \\ }
  \uncover<6-> { z = k + c \cdot x \\ }
  \uncover<7-> { \< \sendmessageright{top=\text{$(m, \sigma=(R, z))$}} \< \\ }
  \< \< \uncover<8-> { c = H(R, Y, m) \\ }
  \< \< \uncover<9-> { R' = g^z \cdot Y^{-c} \\ }
  \< \< \uncover<10-> { \text{Output } R \stackrel{?}{=} R' \\ }
}
}
\end{frame}

\begin{frame}
  \frametitle{FROST Keygen}

  \begin{itemize}
    \item<1-> Can be performed by either a trusted dealer or a Distributed Key
      Generation (DKG) Protocol
    \item[]
    \item<2-> The DKG is an $n$-wise Shamir Secret Sharing protocol, with each
      participant acting as a dealer
    \item[]
    \item<3> After KeyGen, each participant holds secret share $s_i$ and
      public key $Y_i$ (used for verification during signing) with joint public key $Y$.
  \end{itemize}

\end{frame}

\begin{frame}
  \frametitle{FROST Sign}

  \begin{itemize}
    \item<1-> Can be performed in two rounds, or optimized to single round with
      preprocessing
    \item[]
    \item<2-> We show here with a signature aggregator, but can be
      performed without centralized roles
  \end{itemize}
\end{frame}

\begin{frame}
  \centering
\begin{tikzpicture}
\node at (0,0) {
\scalebox{1.0} {
\procedure{FROST Preprocess}{
  \textbf{Participant i}  \< \< \textbf{Commitment Server} \\
    \uncover<1-> { ((d_{ij}, e_{ij}), \dots) \random \mathbb{Z}_q^* \times \mathbb{Z}_q^* \\ }
    \uncover<2-> { (D_{ij}, E_{ij}) = (g^{d_{ij}}, g^{e_{ij}}) \\ }
    \uncover<3-> { \text{Store } ((d_{ij}, D_{ij}), (e_{ij}, E_{ij}), \dots) \\ }
    \uncover<4-> { \< \sendmessageright{top=\text{$((D_{ij}, E_{ij}), \dots)$}} \< \\ }
    \uncover<5-> { \< \< \text{Store } ((D_{ij}, E_{ij}), \dots) \\ }
}
}
};
\node<6> [fill=yellow,draw=red,ultra thick, text width=.4\textwidth,anchor=north west,ellipse callout,text centered,callout
  absolute pointer = {(-4.2,1)}] at (-1,1) { In the two-round variant, this
  step is performed immediately before signing with only one commitment.};
\end{tikzpicture}
\end{frame}

\begin{frame}
  \centering
\begin{tikzpicture}
\node at (0,0) {
\scalebox{0.9} {
\procedure{FROST Sign}{
  \textbf{Signer i}  \< \textbf{Signature Aggregator} \\
  \uncover<1-> {\< B=((1, D_1, E_1), \dots, (t, D_t, E_t)) \\}
  \uncover<2-> { \> \sendmessageleft{top=\text{$(m, B) $}}  \> \\ }
  \uncover<3-> { \rho_\ell = H_1(\ell, m, B), \ell \in S \\ }
  \uncover<4-> { R = \prod_{\ell \in S} D_\ell  \cdot ({E_\ell})^{\rho_\ell} \\ }
  \uncover<5-> { c = H_2(R, Y, m) \\ }
  \uncover<6-> { z_i = d_i + (e_i \cdot \rho_i) + \lambda_i \cdot s_i \cdot c \\ }
  \uncover<8-> { \> \sendmessageright{top=\text{$z_i$}} \> \\ }
  \uncover<9-> { \> \> \text{Publish } \sigma=(R, z=\sum z_i) }
}
}
};
\node<3> [fill=yellow,draw=red,ultra thick, text width=.3\textwidth,anchor=north west,ellipse callout,text centered,callout
  absolute pointer = {(-6.2,0.2)}] at (-1,-1) { ``binding value'' to bind signing shares to $\ell$, $m$, and $B$ };
\node<7> [fill=yellow,draw=red,ultra thick, text width=.3\textwidth,anchor=north west,ellipse callout,text centered,callout
  absolute pointer = {(-4.25,-1.5)}] at (1,1) { This step cannot be inverted by
  anyone who does not know $(d_i, e_i)$.};
  \node<7>[rectangle, draw=red, line width = 1.5pt,
  minimum width = 2cm, minimum height =.65cm] at (-5,-1.85) {};
\node<10> [fill=yellow,draw=red,ultra thick, text width=.3\textwidth,anchor=north west,ellipse callout,text centered,callout
  absolute pointer = {(3.2,-3.2)}] at (1,1) { Signature format and
  verification are identical to single-party Schnorr.};
\end{tikzpicture}
\end{frame}

\begin{frame}
  \frametitle{Security against Drijvers}
  \small

  \begin{itemize}
    \item<1->[] Without $\rho_\ell = H_1(\ell, m, B) $,
  an adversary could produce a $c^*$ such that:
      \[ c^*= H(R^*, Y, m^*) \uncover<2->{ = \sum_{i=1}^{k} H(R_i, Y, m_i)  =
      \sum c_i  \text{ for some } (R_i, m_i), \dots } \],
    \item<3->[] After sending receiving the victim's $z_i$ for each $(R_i, m_i)$, the
  adversary can produce a valid forgery $\sigma^*=(R^*, z)$, as
      \[ \uncover<4-> {z = \sum d_i + e_i + \lambda_t \cdot s_t \cdot \sum c_i} = \uncover<5->
      { \sum d_i + e_i + \lambda_t \cdot s_t \cdot c^*} \]
    \item<6->[] The binding factor in FROST makes each $z_i$ strongly tied to
      $(m_i, R_i)$.
      \[ z = \sum d_i + (e_i * \rho_i) + \lambda_t \cdot s_t \cdot \sum c_i  \]
    \item<7->[] Resulting in an invalid signature: $R^* \neq g^z \cdot Y^{-c}$
  \end{itemize}
\end{frame}

\begin{frame}
  \frametitle{Protocol Complexity}

  \begin{itemize}
    \item<1-> Per-signer bandwidth overhead for signing scales linearly
      relative to the number of signers (because of $B$).
    \item[]~
    \item<2-> Total bandwidth overhead scales quadratically.
    \item[]~
    \item<3-> Network round complexity remains constant, assuming centralized
      commitment storage and signature aggregation.
  \end{itemize}

\end{frame}

\begin{frame}
  \frametitle{Security}

  \begin{itemize}
    \item<1-> We prove the EUF-CMA security of an interactive variant of FROST,
      then extend to plain FROST.
    \item[]
    \item<2-> FROST-Interactive generates the
      binding value $\rho_i$ via a one-time VRF to allow for parallelism in our
      simulator.
    \item[]
    \item<3->Recall that plain (non-interactive) FROST generates $\rho_i$ via a
      hash function.
  \end{itemize}
\end{frame}

\begin{frame}
  \frametitle{Real-World Applications}

  \begin{itemize}
    \item<1-> Planned to be used in cryptocurrency (Zcash) protocols for
      signing financial transactions.
    \item[]
    \item<2-> Under consideration for standardization by CFRG.
    \item[]
    \item<3-> Presented FROST at the recent NIST workshop on standardizing
      threshold schemes.
  \end{itemize}
\end{frame}

\begin{frame}
  \frametitle{Takeaways}
  \begin{itemize}
    \item<1-> FROST improves upon prior schemes by defining a
      single-round threshold signing protocol (with preprocessing) that is
      secure even when signing is performed concurrently.
    \item[]
    \item<2-> The simplicity and flexibility of FROST makes it attractive to
      real-world applications.
  \end{itemize}
      \let\thefootnote\relax\footnote{
      Find our paper and artifact at
      \url{https://crysp.uwaterloo.ca/software/frost}.
      }
\end{frame}

\begin{comment}
\begin{frame}
  \huge
  \centering
  EXTRAS
\end{frame}
\end{comment}


\end{document}


