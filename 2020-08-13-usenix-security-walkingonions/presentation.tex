\documentclass{beamer}
\usetheme{Boadilla}

\usepackage{amsthm, amsfonts, amssymb, amsmath, graphicx}
\usepackage{amsmath}
\usepackage{amsfonts, txfonts}
\usepackage{wasysym}

\usepackage{graphicx}
\usepackage{tikz}

\pgfdeclareimage[height=3ex]{crysplogo}{crysplogo}
\pgfdeclareimage[height=3ex]{torlogo}{torlogo}

\titlegraphic{
\raisebox{1ex}{
\pgfuseimage{crysplogo}~~~
\pgfuseimage{torlogo}
}
}


\title{Walking Onions}
\subtitle{Scaling Anonymity Networks while Protecting Users}
\author[Chelsea Komlo]{Chelsea H. Komlo\\{\footnotesize
Cryptography, Security, and Privacy (CrySP) Research
Group, University of Waterloo}}
\institute{\small Joint work with Nick Mathewson, Ian Goldberg}
\date{\small USENIX Security, 13 August, 2020}

\begin{document}
\setbeamertemplate{itemize items}[triangle]

\begin{frame}
        \thispagestyle{empty}
        \maketitle
\end{frame}


\begin{frame}
\frametitle{Motivation}
  \begin{itemize}
    \item Design protocols that enable anonymity networks like Tor to scale
      significantly beyond their current sizes.
        \begin{itemize}
          \item Add relays to network to increase capacity
          \item While not increasing the overhead to clients to participate.
        \end{itemize}
    \item Do all of this in a way that does not change their existing security model
  \end{itemize}
\end{frame}

\begin{frame}
\frametitle{Our Contributions}
  \begin{itemize}
    \item Walking Onions, a set of protocols that allows anonymity networks
      like Tor to scale with constant-size overhead to users.
    \item Walking Onions maintains the existing security model for Tor in one
      variant, with a slight loss of forward secrecy in the other variant.
    \item Demonstrates performance improvements at networks the size of Tor
      today.
  \end{itemize}
\end{frame}

\begin{frame}
\frametitle{Outline}
  \begin{enumerate}
    \item Review of Current Tor and Performance Bottlenecks
    \item Requirements to maintain similar security
    \item Walking Onions
    \item Comparison to current Tor
      \begin{enumerate}
        \item Security
        \item Performance
      \end{enumerate}
  \end{enumerate}
\end{frame}

\begin{frame}
  \centering
  \huge
  Current Tor
\end{frame}

\begin{frame}
\frametitle{Current Tor Security Model}

Current Tor optimizes for security over scalability

  \begin{itemize}
    \item \textbf{Epistemic Attacks}
    \item \textbf{Route-Capture Attacks}
  \end{itemize}

\end{frame}

\begin{frame}
\frametitle{Current Tor Scalability}

Quadratic cost to client relative to number of relays.

\end{frame}

\begin{frame}
  \centering
  \huge
  Walking Onions
\end{frame}

\begin{frame}
\frametitle{Vanilla Walking Onions}
\end{frame}

\begin{frame}
\frametitle{Telescoping Walking Onions}
\end{frame}

\begin{frame}
\frametitle{Why do these maintain Tor's existing security level?}
\end{frame}

\begin{frame}
\frametitle{}

\begin{table}[t]
\renewcommand{\arraystretch}{1.2}
\caption{Comparison: Telescoping, Single-Pass, Current Tor }

\centering
\footnotesize

\CIRCLE=achieved; \Circle=not achieved;
  \LEFTcircle=partially achieved

$\Diamond$=performance property;
  $\dagger$=security property

    \begin{tabular}{|c|c|c|c|c|}
  \hline
  & & Telescop. & Single-Pass & Current Tor \\
  \hline
  $\Diamond$ & Constant-size client download & \CIRCLE & \CIRCLE & \Circle \\
  \hline
  $\Diamond$ & One round trip per circuit built & \Circle & \CIRCLE & \Circle \\
  % Start security properties
  \hline
  $\dagger$ & \raggedright Complete client control of relays selected & \LEFTcircle & \Circle & \CIRCLE \\
  \hline
  $\dagger$ & Forward-secret relay selection& \CIRCLE & \LEFTcircle & \CIRCLE \\
  \hline
  $\dagger$ & Forward secrecy for data & \CIRCLE & \CIRCLE & \CIRCLE \\
  \hline
  $\dagger$ & \raggedright Relays unaware of their positions in paths & \LEFTcircle & \Circle & \LEFTcircle \\
  \hline
\end{tabular}

\end{table}
\end{frame}


\begin{frame}
  \centering
  \huge
  Performance Evaluation
\end{frame}

\begin{frame}
\frametitle{Simulation}
Simulation description
\end{frame}

\begin{frame}
\frametitle{Bandwidth}
\end{frame}

\begin{frame}
\frametitle{CPU}
\end{frame}

\begin{frame}
\frametitle{Summary}
\end{frame}
\end{document}
