% https://nymity.ch/tikz/

\documentclass[hyperref={pdfpagelabels=true},table,dvipsnames,14pt,aspectratio=169]{beamer}
\usetheme{Boadilla}
\setbeamertemplate{navigation symbols}{}

\usepackage{amsthm, amsfonts, amssymb, amsmath, graphicx}
\usepackage{amsmath}
\usepackage{mathtools}
\usepackage{amsfonts, txfonts}
\usepackage{wasysym}
\usepackage{xcolor}
\usepackage{caption}
\usepackage{comment}
\usepackage{cryptocode}
\urlstyle{same}
\usepackage{framed}
\FrameSep5pt

\usepackage{graphicx}
\graphicspath{ {images/} }

\usepackage{tikz}
\usepackage{tikzpeople}
\usepackage{pgfplots}
\usetikzlibrary{arrows}

\usetikzlibrary{positioning}
\usetikzlibrary{calc}
\usetikzlibrary{shapes}

\usepackage{capt-of}
\beamertemplatenavigationsymbolsempty
\setbeamercovered{transparent}

\newcommand{\sigagg}{\mathcal{SA}}
\newcommand{\random}{\overset{\$}{\leftarrow}}

% Method to add an item that will appear in every slide
%\addtobeamertemplate{footline}{%
%
%  \tikz[remember picture, overlay] \fill[blue] (current page.south east)
%    rectangle ++(1cm,2cm);
%
%}{}



\pgfdeclarelayer{background}
\pgfdeclarelayer{backbackground}
\pgfdeclarelayer{foreground}
\pgfsetlayers{backbackground,background,main,foreground}   %% some additional layers for demo

\usetikzlibrary{shapes,decorations.shapes}
\tikzset{>=latex}


\title[Threshold Signatures]{Schnorr Threshold Signatures  \\[1em] \normalsize{An Overview of the Current Landscape and Next Steps}}
\author{Chelsea Komlo}
\institute[]{\small University of Waterloo}
\date[August 2021]{ \small Microsoft Privacy and Cryptography Group, August 2021}

\begin{document}
\setbeamertemplate{itemize items}[triangle]

\begin{frame}
        \thispagestyle{empty}
        \maketitle
\end{frame}


\begin{frame}
  \frametitle{Threshold Signatures: Joint Public \\ Key, Secret-Shared Private Key}

\begin{tikzpicture}
      \begin{pgfonlayer}{foreground}
    \node<2> [fill=yellow,draw=red,ultra thick, text width=.3\textwidth,anchor=north west,ellipse callout,text centered,callout
  absolute pointer = {(9.2,1)}] at (12,1) { The full secret
        is never reconstructed! Participants perform signing using only their secret
        share.};
      \end{pgfonlayer}
      \begin{pgfonlayer}{main}
      \node[person,shirt=black,female, minimum size=1.1cm] (p2) at (7.5,1) {
        $\includegraphics[width=.075\textwidth]{images/seckey.png}_\text{\bf
        Secret Share 1}$
      };
      \node[person,shirt=black, female, minimum size=1.1cm] (p6) at (9,-2) {
        $\includegraphics[width=.075\textwidth]{images/seckey.png}_\text{\bf
        Secret Share 2}$
      };
      \node[person,shirt=black,female, minimum size=1.1cm] (p7) at (13,1) {
        $\includegraphics[width=.075\textwidth]{images/seckey.png}_\text{\bf
        Secret Share 3}$
      };

      \end{pgfonlayer}

      \begin{pgfonlayer}{backbackground}
      \node[circle, minimum size = 5.4cm, color=lightgray,
        fill=lightgray]  (circ) at (10.5,-0.5){};
      \end{pgfonlayer}
      \begin{pgfonlayer}{background}
          \node[ellipse, draw=red, line width = 2pt,
          minimum width = 9.25cm, minimum height =3cm] at (10,0.5) {\small
          \bf Signing Set};
      \end{pgfonlayer}

       \node[outer sep = 2mm] (pubkey) at (18,-0.5)
       {
         $\includegraphics[width=.15\textwidth]{images/pubkey.png}_\text{\bf
         Public Key (1,2,3)}$  };

\end{tikzpicture}

\end{frame}

\begin{frame}
  \centering
\scalebox{0.9} {
\procedure{Single-Party Schnorr Signing and Verification}{
  \textbf{Signer}  \< \< \textbf{Verifier} \\
  \uncover<1-> { (x, Y) \leftarrow \mathit{KeyGen()} \\ }
  \uncover<2-> { \< \sendmessageleft{top=\text{$(m, Y)$}} \< \\ }
  \uncover<3-> { k \random \mathbb{Z}_q} \\
  \uncover<4-> { R = g^k \in \mathbb{G} } \\
  \uncover<5-> { c = H(R, Y, m) \\ }
  \uncover<6-> { z = k + c \cdot x \\ }
  \uncover<7-> { \< \sendmessageright{top=\text{$(m, \sigma=(R, z))$}} \< \\ }
  \< \< \uncover<8-> { c = H(R, Y, m) \\ }
  \< \< \uncover<9-> { R' = g^z \cdot Y^{-c} \\ }
  \< \< \uncover<10-> { \text{Output } R \stackrel{?}{=} R' \\ }
}
}
\end{frame}


\begin{frame}
  \centering
\begin{tikzpicture}
\node at (0,0) {
\scalebox{0.9} {
\procedure{Warm up Example of Threshold Signatures }{
  \textbf{Signer i}  \< \textbf{Signature Aggregator} \\
  \uncover<1-> { d_i \random \mathbb{Z}_q^* \times \mathbb{Z}_q^*  \\ }
  \uncover<2-> { \> \sendmessageright{top=\text{$D_i=g^{d_i}$}}   \\ }
  %
  \uncover<3-> { \> \sendmessageleft{top=\text{$(m,  B=((1, D_1), \dots, (t, D_t)) ) $}}  \> \\ }
  \uncover<4-> { R = \prod_{\ell \in S} D_\ell  \\ }
  \uncover<5-> { c = H_2(R, Y, m) \\ }
  \uncover<6-> { z_i = d_i +  \lambda_i \cdot s_i \cdot c \\ }
  \uncover<7-> { \> \sendmessageright{top=\text{$z_i$}} \> \\ }
  \uncover<8-> { \> \> \text{Publish } \sigma=(R, z=\sum z_i) }
}
}
};
\end{tikzpicture}
\end{frame}



\begin{frame}
  \frametitle{Drijver's Attack}

  \begin{itemize}
    \item<1-> Forgery attack on two-round multisignatures/threshold signatures.
    \item[]
    \item<2-> Requires an active attacker (a signer participating in the protocol).
    \item[]
    \item<3-> Relies on the Wagner-Fischer algorithm for finding a collision between hash function outputs.
    \item[]
    \item<4-> Difficult to find $H(x) = H(y)$.
    \item[]
    \item<5-> But finding an $x$ such that $H(x) = H(w) + H(y) + H(z)+ \ldots$ for some $(w, y, z)$  is possible in
      polynomial time.
  \end{itemize}
\end{frame}

\begin{frame}
  \frametitle{Drijver's Attack against the Warm Up Example}
  \small

  \begin{itemize}
    \item<1->[] An adversary can query an individual signer, producing different
      $D_A, m_A$ terms for themselves (therefore varying $R, c$) each time.
    \item<2->[] Eventually, an adversary could produce a $c^*$ such that:
      \[ c^*= H(R^*, Y, m^*) \uncover<2->{ = \sum_{i=1}^{k} H(R_i, Y, m_i)  =
      \sum c_i  \text{ for some } (R_i, m_i), \dots } \],
    \item<3->[] After sending receiving the victim's $z_i$ for each $(R_i, m_i)$, the
  adversary can produce a valid forgery $\sigma^*=(R^*, z)$, as
      \[ \uncover<4-> {z = \sum d_i +  \lambda_t \cdot s_t \cdot \sum c_i} = \uncover<5->
      { \sum d_i + \lambda_t \cdot s_t \cdot c^*} \]
  \end{itemize}
\end{frame}


\begin{frame}
  \frametitle{Tradeoffs Among Constructions}

  \begin{itemize}
    \item<1-> \textbf{Number of Signing Rounds:} Required network rounds to
      generate one signature.
    \item<2-> \textbf{Robust:} Can the protocol complete when participants
      misbehave?
    \item<3-> \textbf{Required Number of Signers:} Can a signature be
    created by just $t$ participants, or are all $n$ needed?
    \item<4-> \textbf{Parallel Secure:} Can signing operations be done in
      parallel without a reduction in security (Drijvers attack)?
  \end{itemize}
\end{frame}

\begin{frame}
  \frametitle{Tradeoffs Among Constructions}

  \begin{table}[t]
    \centering
    \small
    \begin{tabular}{c|c c c c}
      & \textbf{Num. Rounds} & \textbf{Robust} & \textbf{Num. Signers}&
      \textbf{Parallel Secure} \\
      \hline
      Stinson Strobl & 4 & Yes & $t$ & Yes \\
      \hline
      Gennaro et al. & 1 w/ preprocessing & No & $n$ & No \\
      \hline
      FROST & 1 w/ preprocessing & No & $t$ & Yes \\
    \end{tabular}
\end{table}
\end{frame}

\begin{frame}
  \frametitle{Contributions of FROST \\ \small{Flexible Round-Optimized Schnorr Threshold Signatures}}

  \begin{itemize}
    \item<1-> Two-round threshold signing protocol, or single-round \\ protocol with
    preprocessing
    \item[]
    \item<2-> Secure against the Drijvers attack, for an adversary controlling up to $t-1$ signers.
    \item[]
    \item<3-> Signing can be performed with a threshold $t$ number of signers,
      where $t$ can be less than the number of possible signers $n$.
  \end{itemize}
\end{frame}




\begin{frame}
  \centering
\begin{tikzpicture}
\node at (0,0) {
\scalebox{0.9} {
\procedure{FROST Sign}{
  \textbf{Signer i}  \< \textbf{Signature Aggregator} \\
  \uncover<1-> { (d_i, e_i) \random \mathbb{Z}_q^* \times \mathbb{Z}_q^*  \\ }
  \uncover<2-> { \> \sendmessageright{top=\text{$(D_i=g^{d_i}, E_i=g^{e_i})$}}   \\ }
  %
  \uncover<3-> { \> \sendmessageleft{top=\text{$(m,  B=((1, D_1, E_1), \dots, (t, D_t, E_t)) ) $}}  \> \\ }
  \uncover<4-> { \rho_\ell = H_1(\ell, m, B), \ell \in S \\ }
  \uncover<6-> { R = \prod_{\ell \in S} D_\ell  \cdot ({E_\ell})^{\rho_\ell} \\ }
  \uncover<7-> { c = H_2(R, Y, m) \\ }
  \uncover<8-> { z_i = d_i + (e_i \cdot \rho_i) + \lambda_i \cdot s_i \cdot c \\ }
  \uncover<10-> { \> \sendmessageright{top=\text{$z_i$}} \> \\ }
  \uncover<11-> { \> \> \text{Publish } \sigma=(R, z=\sum z_i) }
}
}
};
\node<5> [fill=yellow,draw=red,ultra thick, text width=.3\textwidth,anchor=north west,ellipse callout,text centered,callout
  absolute pointer = {(-5.5,-0.5)}] at (-1,-1) { ``binding value'' to bind signing shares to $\ell$, $m$, and $B$ };
\node<9> [fill=yellow,draw=red,ultra thick, text width=.3\textwidth,anchor=north west,ellipse callout,text centered,callout
  absolute pointer = {(-4.25,-1.5)}] at (1,1) { This step cannot be inverted by
  anyone who does not know $(d_i, e_i)$.};
  \node<9>[rectangle, draw=red, line width = 1.5pt,
  minimum width = 2cm, minimum height =.65cm] at (-5.75,-2.3) {};
\node<12> [fill=yellow,draw=red,ultra thick, text width=.3\textwidth,anchor=north west,ellipse callout,text centered,callout
  absolute pointer = {(3.2,-3.2)}] at (1,1) { Signature format and
  verification are identical to single-party Schnorr.};
\end{tikzpicture}
\end{frame}

\begin{frame}
  \frametitle{Security against Drijvers}
  \small

  \begin{itemize}
    \item<1->[] Without $\rho_\ell = H_1(\ell, m, B) $,
  an adversary could produce a valid forgery $\sigma^*=(R^*, z)$, as
      \[ \uncover<2-> {z = \sum d_i + e_i + \lambda_t \cdot s_t \cdot \sum c_i} = \uncover<3->
      { \sum d_i + e_i + \lambda_t \cdot s_t \cdot c^*} \]
      \[ R^* = g^{\sum d_i + e_i + s \cdot \sum c_i     } \cdot g^{-s c^*}\]
    \item<4->[] The binding factor in FROST makes each $z_i$ strongly tied to
      $(m_i, R_i)$.
      \[ z = \sum d_i + (e_i * \rho_i) + \lambda_t \cdot s_t \cdot \sum c_i  \]
    \item<5->[] Resulting in an invalid signature:
      \[ R^* \neq g^z \cdot Y^{-c}\]
      \[ R^* \neq g^{\sum d_i + (e_i * H(m, (D_i, E_i, i), \ldots)) + s \cdot \sum c_i     } \cdot g^{-s c^*}\]
  \end{itemize}
\end{frame}


\begin{frame}
  \frametitle{Implementation Requirements}

  \begin{itemize}
    \item<1-> KeyGen requires a trusted, authenticated channel for distributing
      secret shares.
    \item[]
    \item<2-> Signing can be performed over a trustless public channel as all
      values exchanged during signing are public.
    \item[]
    \item<3-> Use of some underlying PKI is required for proving attribution of
      misbehaviour to a specific signer.
  \end{itemize}
\end{frame}



\begin{frame}
  \frametitle{Takeaways}
  \begin{itemize}
    \item<1-> FROST improves upon prior schemes by defining a
      single-round threshold signing protocol (with preprocessing) that is
      secure even when signing is performed concurrently.
    \item[]
    \item<2-> The simplicity and flexibility of FROST makes it attractive to
      real-world applications.
  \end{itemize}
      \let\thefootnote\relax\footnote{
      Find our paper and artifact at
      \url{https://crysp.uwaterloo.ca/software/frost}.
      }
\end{frame}




\end{document}


